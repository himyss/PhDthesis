\section{Experiment}

This work presents the results of several experiments, dedicated to the $^{6,7}$H studies and one reference measurement of the same reaction mechanism, performed to control all the setup parameters and to test the reliability of the obtained results.
All the experiments were conducted at the ACCULINNA-2 fragment separator \cite{Fomichev:2018}, recently constructed in the Flerov Laboratory of Nuclear Reaction of the Joint Institute for Nuclear Research. 
In this chapter, one briefly presents the main features of the ACCULINNA-2 facility and describes in more details the experimental setups, employed for the investigations of the isotopes of interest.

\subsection{ACCULINNA-2}

The idea of the ACCULINNA-2 in-flight separator is to provide, separate and transport the low-energy (from 10 to 40\,AMeV) RIBs of high intensity and purity. 
One should not, that it is not appropriate to compare its properties with those of such large facilities as FRS \cite{geissel:1992} and SuperFRS \cite{geissel:2003,winkler:2008} at FAIR, ARIS at FRIB \cite{gao:2015}, BigRIPS at RIKEN \cite{kubo:2003}, or others \cite{cuttone:2007,www:ganil,www:isolde}.
The scientific uniqueness of the ACCULINNA-2 facility is illustrated in Fig.\ \ref{fig:acculinna2_worldplace}.
One may see, that among all other in-flight separators, the chosen one is able to provide the light RIBs of the lowest possible energy, which is indispensable for certain experimental issues.
The technical approach of the facility allows one to provide a high energy and angular resolution of the secondary beams, which in conjunction with high efficiency for correlation measurements allows user to analyse multiple kinematic conditions of the projectile and the reaction products.  
That is why, selection of the studied reaction channel can be easily carried out, which, afterwards, leads to high quality of the spin-parity identification for the excitation spectra of the systems of interest.

%-------------------------------------------------------------------------------
\begin{figure}[t]
	\begin{center}
		\includegraphics[width=0.7\textwidth]{figures/acculinna2_part.png}
	\end{center}
	%
	\caption{
		Landscape of the present-day facilities on the diagram where for radioactive beams, specified in terms of their atomic numbers, the available RIB energy ranges are shown.}
	%
	\label{fig:acculinna2_worldplace}
\end{figure}
%-------------------------------------------------------------------------------

The ACCULINNA-2 facility is coupled to the U400-M cyclotron, which provides high intensity primary beams of $^{7}$Li, $^{11}$B, $^{13}$C, $^{15}$N and $^{18}$O of energies  between 30 and 50\,AMeV.
Of course, the fragment separator can be configured in a mode to form the mentioned stable beams and deliver them into the reaction chamber.  
The ACCULINNA-2 facility is 36 meters long achromatic separator consisting of two 45-degree dipole magnets, 14 quadrupoles, eight multipoles (three octupoles and five sextupoles) and four steering magnets, see Fig.\ \ref{fig:acculinna2_scheme}.
The high intensity primary beams are delivered by the U-400M cyclotron to the rotating production target module installed in the first intermediate focal plane F1 to produce radioactive ion beams in fragmentation reactions via the in-flight method.

The production target module, integrates a vacuum chamber with a water cooled beryllium target mounted on rotating magnetic liquid feed through and a set of water-cooled diaphragms. 
The target module is designed to work with heating power up to 2\,kW. 
Radioactive nuclei leaving the production target are captured by a short-focusing quadrupole triplet Q1–Q3 and are transported through the magnetic dipoles D1–D2 and magnetic quadrupoles Q4–Q14 up to the final focal plane F5. 
Magnetic multipoles with corresponding sextupole and octupole components are used for correction of second- and third- order aberrations occurring otherwise in the F2 and F3 planes (see in Fig.\ \ref{fig:acculinna2_scheme}). 

The purification of the reaction fragments is achieved by a separation method based on magnetic-rigidity analysis and energy-loss in a degrader material. 
The D1 dipole magnet filters fragments by their magnetic rigidity $B$ , providing dispersion at the focal plane F2. 
The relation between the magnetic rigidity of the first bending magnet and the $A/Z$ number is given by $B=p/q$. 
Further purification is achieved by separation of the fragments by their energy losses in the wedge-shaped degrader. 
The second dipole D2 compensates the dispersion occurring in the focal plane F2 and collects the fragments at the achromatic focal plane F3. 
Identification of the reaction products is performed by the measurement of the Time of Flight (TOF) and energy
loss ($E$) of the fragments.
Both of them are measured by two BC404 plastic scintillation detectors installed in the the F3 and F5 focal planes with 12.3-meter base, see in Fig.\ \ref{fig:exp-common}.
Each scintillator is coupled with four Hamamatsu R7600-200 photomultiplier tubes (PMT). 
With these devices one can reach the time resolution of 100\,ps, which determines the accuracy of determining of the beam kinetic energy.
The selection made for the thicknesses of the production target and wedge-shaped degrader in accord with the position determination made for the momentum slits standing in focal planes F2–F5 have direct influence on the yield and purity of the required RIB.
This is typically sufficient for the production of quite pure RIBs of light, neutron-rich exotic nuclei. Proton-rich RIBs need additional purification from a large amount of contamination. 
The reason of that is the fragmentation mechanism leads to the low-energy tails of obtained in the energy spectra of the very well-produced undesirable less proton-rich nuclei. 
Adding the velocity separation to the magnetic-rigidity analysis one can drastically reduce the effect of this contamination. 
For this purpose a radio-frequency (RF) kicker is installed in on the beam-line between the F3 and F4 focal planes (see in Fig.\ \ref{fig:acculinna2_scheme}). 

%-------------------------------------------------------------------------------
\begin{figure}
	\begin{center}
		\includegraphics[width=1\textwidth]{figures/exp-common}
	\end{center}
	%
	\caption{
		Experimental setup common for the experiments 1 and 2 at the final focal plane F5 of the ACCULINNA-2 fragment separator.
		Detectors of the beam projectile positions (MWPC-1,2) and time-of-flight (ToF) are described in the text.
		There is no common scale along the beam axis in this plot.	
		The telescope detectors of the reaction products are also shown in Figs.\ \ref{fig:setup-1} and \ref{fig:setup-2}.
	}
	%
	\label{fig:exp-common}
\end{figure}
%-------------------------------------------------------------------------------

The secondary beam tracking is realized by pair of multi-wire proportional chambers (MWPC), illustrated in the sketch in Fig.\ \ref{fig:exp-common}.
These detectors were placed at the distances of 28 and 81\,cm upstream of the experimental target plane, located in the reaction chamber.
The MWPCs are filled with CF$_{4}$ and CH$_{4}$ gas mixture with a ratio 9 to 1 and atmospheric pressure.
Each detector consists of two layers, 32 wires each, installed with interval of 1.25\,mm. 
This allowed to determine the RIB interaction points in the target plane with a 1.8\,mm  precision.
Also, this beam-tracking installation determined the inclination angles of individual RIB projectiles to the ion optical axis with an accuracy of $\approx 0.15 $\,degrees.


%-------------------------------------------------------------------------------
\begin{figure}[t]
	\begin{center}
		\includegraphics[width=1\textwidth]{figures/acculinna2.png}
	\end{center}
	%
	\caption{Lay-out of the fragment-separator ACCULINNA-2. F1 – the object plane; F2 – the intermediate dispersion plane; F3, F4 – the achromatic focal planes; F5 – the final focal plane.}
	%
	\label{fig:acculinna2_scheme}
\end{figure}
%-------------------------------------------------------------------------------

The desired beam (RIB or primary), passing through the described beam diagnostics system, reaches the stainless steam vacuum reaction chamber, located in the experimental room.
The latter is well shielded by 2\,m thick concrete wall in order to get rid of the background from the cyclotron.
The created conditions inside the reaction chambers are allowed to use cryogenic gaseous targets (even tritium) and all the modern detector types.
Moreover, the space inside the experimental hall allows to use such large facilities as zero angle spectrometer (recently installed) or neutron detection walls.
All described makes the ACCULINNA-2 facility a powerful tool for research made in the fields of light exotic nuclei near the nucleon stability borders.

\subsection{Experimental setup}

All the experiments were performed in so-called inverse kinematics, which means that the studied nuclear reactions are induced by a heavy RIB projectile colliding with lighter target nuclei.
The technique of using reversed kinematics is costly in terms of the available center-of-mass energy, however, it significantly simplifies the experimental setup.
In such conditions, most of the projectile's energy goes into forward motion of the reaction products in the laboratory system. 
That is why, mostly, all the reaction products move forward in relatively narrow cone along the secondary beam direction in laboratory frame, which in most cases, allows to detect all reaction products with reasonable efficiency, and therefore, to study the occurred reaction at any angular ranges. 

It is well known, that in direct transfer processes, the reaction cross section has a maximum at very forward angles \textcolor{red}{[???]}. 
The concept of both experiments on heavy hydrogen isotopes investigations was to archive the record statistics by measuring the reaction products at the most possible forward angles.
One can find out, that this idea was one of the motivations to modify the detector system after the first experimental run, see the detector's systems schemes, described below.

%===============================================================================

\subsubsection{Experiment 1}
\label{sec:exp-1}

%===============================================================================

The first experiment was performed in 2018 with use of the $^{8}$He beam with purity of $\approx90\%$ and intensity of $\approx10^{5}$\,pps and the energy in the middle target plane spread within $\pm7$ around the mean value of 26\,AMeV.
This secondary beam was produced by impinging of the $^{11}$B primary beam ($\approx 1$ p$\mu$A, 33.4\,AMeV) with  the 1-mm thick beryllium production target.
The target cell, 25\,mm in diameter and 4-mm thick, was filled with deuterium gas (further D$_{2}$) at temperature of 30\,K and atmospheric pressure. 
In order to keep the gas inside the cell, target was equipped with 6\,$\mu$m thick stainless-steel entrance and exit windows.
The cell was also concealed in a screened volume having a pair of 3.5\,$\mu$m thick aluminum-backed Mylar windows and kept cooled to the same temperature to ensure thermal protection.
The entrance/exit target windows, deformed by the gas pressure, took the near-lenticular form, so that the maximum target thickness turned out to be 6\,mm.
In the described conditions, the D$_{2}$ target thickness was $\approx 3.8 \times 10 ^{20}$ cm$^{-2}$.
To ensure a homogeneous thickness of the target, only events when the RIB hit a central part of the target with a circular shape of the diameter of 17\,mm were taken into account.
This selection ensured also the rejection of the reactions with the material of the target frame, made of stainless-steel.

For detection of the low energy recoil particles, emitted from the target, two identical $\Delta E$-$E$-$E$ single-sided silicon-detector telescopes were used. 
The latter allowed to measure the energy and to reconstruct the emission angle of the passing particles with accuracy \textcolor{red}{$\approx1\%$} and $\approx1^{\circ}$ respectively.
Therefore, these assemblies provided high quality identification and tracking of charged particles with energies from few MeV up to $100$\,MeV.
The described side telescopes were located 166\,mm downstream from the target and covered angular range between $8^{\circ}$ and $26^{\circ}$ in the laboratory system, see Fig.\ \ref{fig:setup-1} 
Each telescope consisted of three layers of SSDs. 
A 20-$\mu$m thick Si $\Delta$E-SSD of model W1 (SS)-20 Type 2M, produced by Micron Semiconductor Ltd \textcolor{red}{[micron semiconductor website]}, had a sensitive area of $50 \times 50$ mm$^2$, divided into 16 strips, and occupied the front position in the telescopes.
The pair of identical 1-mm thick ($60 \times 60$ mm$^2$ with 16 strips, model \textcolor{red}{???}) SSDs were placed on the second and third positions.
In order to track the particles with such telescopes, the strips of the first (20-$\mu$m) and the second layer detectors were set to be perpendicular to each other. 
The third SSD was used as a veto detector for to eliminate signals coming from the particles penetrating the second layer.

%The energy values of the individual RIB projectiles were determined with precision of $ \approx 0.2 \%$ by means of the Time-of-Flight (ToF) detector. 
%The granularity of the $^{3}$He telescopes
%-------------------------------------------------------------------------------
\begin{figure}
	\begin{center}
		\includegraphics[width=1\textwidth]{figures/setup-1}
	\end{center}
	%
	\caption{Charged particle detector telescopes used in the experiment 1.}
	%
	\label{fig:setup-1}
\end{figure}
%-------------------------------------------------------------------------------

Detection of the high energy particles at zero angles was realized by the central telescope, installed at the beam line at the distance of 280\,mm behind the target and covered angles $\leq9^{\circ}$ in the laboratory system.
For example, in the proton transfer reaction of the $^7$H population, it was intended to detect high energy tritons emitted from $^7$H.
This central telescope consisted of one 1.5\,mm thick double-sided SSD ($64 \times 64$ mm$^2$, with 32 strips on each side) followed by a square array of 16 CsI(Tl) crystals.
The crystals had a cross section of $16.5\times16.5$ mm$^2$ and thickness 50\,mm each, which allows to stop the beam and all charged reaction products in the sensitive volume of the telescope.
Each crystal was covered with a 3.5\,$\mu$m-thick aluminized Mylar on its entrance and was coupled with its Hamamatsu R9880U-20 photomultiplier tube (PMT) by the optical grease.
In order to increase the collection of light and to avoid light cross-talks, each crystal was wrapped in a 100\,$\mu$m-thick VM-2000 reflector.

%===============================================================================

\subsubsection{Experiment 2}
\label{sec:exp-2}

%===============================================================================

The second experiment was dedicated to studies of the $^2$H-$^3$He reaction mechanism, realized with the $^8$He beam of the same properties as in experiment\,1, and the beam of $^{10}$Be with energy of $42\pm7$\,AMeV.
The latter was used for the reference reaction studies in order to test the reliability of the obtained experimental data and to control calibration parameters and all setup configurations.
These measurements were performed just after the main run with $^8$He beam and all conditions related to the experimental setup (experiment\,2) were kept the same.
The most important task of the second experiment was to increase the statistics obtained in the first one and to expand the measured angular range of the recoils to lower values in the laboratory system.
For this purpose, the detector setup was modified, see Fig.\ \ref{fig:setup-2}.
The new side telescope assembly was installed at a distance of 179\,mm from the target.
It consisted of four identical $\Delta E$-$E$-$E$ telescopes made of the same SSDs as described in Section \ref{sec:exp-1}.
The angular range covered by these telescopes for the $^3$He recoil nuclei was extended up to the range from $\approx 6^{\circ}$ to $\approx 24^{\circ}$ in the laboratory system.

The high energy particles, emitted at low angles, were measured in experiment\,2 within more narrow cone in comparison with experiment\,1.
For this purpose, the central telescope, the same as in experiment\,1, was placed at a distance of 323\,mm downstream from the target.
As a result of the made modifications, considered together with the fact that the beamtime of the second experiment was $\approx2$ times longer, we could expect to increase the statistics of the reactions of interest by a factor of $\approx 2.5$.

%-------------------------------------------------------------------------------
\begin{figure}
	\begin{center}
		\includegraphics[width=1\textwidth]{figures/setup-2}
	\end{center}
	%
	\caption{Charged particle detector telescopes used in the experiment 2.}
	%
	\label{fig:setup-2}
\end{figure}
%-------------------------------------------------------------------------------

The setup of the second experiment also included the neutron spectrometer, made of 48 organic scintillator modules \cite{Bezbakh:2018}.
The spectrometer detects neutrons by measuring the light produced by the interaction of the recoil charged particles (mainly protons) within the scintillator.
It was located at zero angle in approximately 2 meters behind the reaction chamber.
The distance between the neighboring modules was approximately 12\,cm, which allowed to cover most of the forward angles, see Fig.\ \ref{fig:exp-common}.
The sensitive part of each module was cylinder made of stilbene monocrystal, C$_{14}$H$_{12}$. 
Each cylinder had 8\,cm diameter and 5\,cm thickness and was oriented by its axis to the target.
Each crystal, covered with reflective MgO powder, was inserted into the 0.5\,mm thick aluminum housing and connected to the PMT by the glass window and optical grease.
Two types of PMT were used: Philips Photonics XP 4312 and ET-Enterprise 9822B. 
In order to decrease the background signals produced by charged particles or $\gamma$-rays, PMT-crystal systems were put into the steel tubes with 0.5\,mm entrance windows.

\subsection{Data acquisition system}

The amplifired signals come from all silicon and scintillation detectors were split into two channel branches and processed by either constant-fraction or leading-edge discriminator. 
The used readout system allowed to use both time-to-digital converter (TDC) and amplitude-to-digital converter (ADC) for all channels of the detector system.
Moreover, we should note that, the signals, corresponded to the neutron spectrometer, were also processed by the time-to-amplitude converter, which provides the data for neutron identification \cite{Bezbakh:2018}.

We used the same logic of the data acquisition, trigger formation and event building, for all conducted experiments.
In order to fulfil both functions of beam diagnostic and selection of the desired events, two different triggers with "OR" coincidence logic, were used for the data acquisition.
In the next chapters, we refer them as "beam" and "experimental" triggers.
If both of them were fired, the system saves the experimental trigger value.
The beam trigger corresponded to those events, when any particle interacted with the ToF scintillation detector, located at F5 focal plane. 
Obviously, one can expect, that the beam trigger fires significantly more often than the experimental one.
Moreover, the data acquisition system has its own processing time, called "busy" status, during which no new coming data can be collected.
That is why the beam trigger was connected to the scaler block, which allowed to reduce the count rate of the corresponded events by a factor of 1024.
The experimental trigger signal was formed by the signals coming from the second layer of the side silicon telescopes.
If any of the trigger signals was fired, and the data acquisition system is not busy, the data from all connected blocks was read out by a frontend VME controller RIO4, controlled by the Multi-Branch System (MBS) data acquisition software [\textcolor{red}{????}].
The written data, consisted of more than 500 signal parameters, had the event structure and stored into binary files.
