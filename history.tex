\section{Light neutron-rich exotic nuclear studies}

Search for limits of existence of unbound neutron-rich systems is one of the main trends in modern nuclear physics.
Probably, the most significant recent results in this field are works on $^{7}$H, $^{10}$He \cite{Sidorchuk:2012,Kohley:2012,Jones:2015,Matta:2015}, $^{13}$Li \cite{Johansson:2010a,Kohley:2013b}, $^{16}$Be \cite{Spyrou:2012}, $^{21}$B \cite{Leblond:2018}, $^{26}$O \cite{Kohley:2013,Caesar:2013,Kondo:2016} and the on-going quest for $^{18}$Be, $^{28}$O, $^{33}$F \cite{Ahn:2019} (and analogous very exotic species). 
One can find that these experiments require extreme technical efforts, and nevertheless, often suffer from poor data quality, characterized by low statistics, resolution and consequently numbers of unresolved data interpretation questions.
One can also find conflicting results, related to the same exotic nuclear system, see, e.g.\ heavy helium cases in review works \cite{Grigorenko:2016,Fortune:2018}
The typical feature of such systems, close to the neutron dripline, is the multineutron (at least two-neutron) emission.
And of course, among all the neutron rich nuclear resonances, 6,7H being the isotopes of the lightest chemical element, draw the attention of physicists because of their biggest neutron-to-proton ratio which can be imagined.

At the present moment, the $^{7}$H system is closest to the neutron matter, which makes it the key isotope of among all neutron rich systems. 
Although, the first prediction of this system existence was made more than 50 years ago \textcolor{red}{[КАКАЯ ТО ССЫЛКА]}, the problem still remains relevant.
The first theoretical estimations were performed by simple quasi-classical model in \cite{Baz:1972} and predicted that the $^{7}$H nucleus could be even bound.

Few experiments, dedicated to search for long-lived $^{7}$H were conducted.
One should mention the attempts to produce this system in the pion double charge exchange reactions, described in works \cite{Seth:1981,Evseev:1981}.
Although such technique was effective to populate and identify the ground states of $^{8,9}$H, it only allowed to obtain the limits of the cross section for both $^{5}$H and $^{7}$H, which differed from each other by the factor of 30.
Because the reconstructed $^{7}$H excitation energy spectrum in \cite{Seth:1981} was very smooth, featureless and had no identifiable bump anywhere, it was an object of multiple speculative interpretations.
But on the other hand, it was in remarkable agreement with the phase space results for $^{5}$H+n+n breakup.
Such observation could be an indication of the correlation $^{5}$H and $^{7}$H systems and hence led to suggestion to study $^{5}$H first.
Even much later, another sophisticated approach was used in the work \cite{Golovkov:2004}, dedicated to search the long-lived quasistable $^{7}$H.
Speculative suggestions about the $^{7}$H decay energies and lifetime led to the assumption of possible very low (50-100\,keV) decay energy level existence.
The estimated lifetime of such state would exceed 1\,ns, and therefore could be measured directly.
The experiment was conducted with a beam of 20.6\,AMeV $^{8}$He and very thick (5.6 cm) liquid deuterium target.
The $^{7}$H nuclei was supposed to produce in the $^2$H($^8$He,$^7$H)$^3$He reaction within $0^{\circ}-50^{\circ}$ c.m.\ angular range, and to measure by $\Delta E$-$E$ detector assembly, located behind the target. 
The absence of $^7$H events with lifetime gave a very low limit for the cross section of the $^2$H($^8$He,$^7$H)$^3$He reaction, $\sigma < 3$ nb/sr, which is by several orders of the magnitude less than the expected value.
The lifetime estimates made in Ref.\ \cite{Golovkov:2004} led to the conclusion that the obtained limit of the $^{7}$H production cross section implies a lower limit of $E_T \gtrsim 50-100$ keV for its decay energy.
This indicates that the only realistic approach to the $^7$H problem is the search for the shorter-lived resonance states of this nucleus in the five-body $^{3}$H+$4n$ continuum.
Also, the experiment \cite{Aleksandrov:1982} aimed to detect this nucleus among the ternary fission products of $^{252}$Cf gave negative result.

The observation of the ground state resonance in $^{5}$H \cite{Korsheninnikov:2001} revived theoretical interest to the possible existence of a low-lying $^{7}$H  state near the $^{3}$H+$4n$ decay threshold.
Based on the known systematic for helium isotopes, so-called helium anomaly, and similar behavior of the hydrogen isotopes, the authors made a suggestion, that $^{7}$H can exists as a resonance, low lying above the decay threshold.   
The mentioned anomaly, illustrated in Fig.\ \ref{fig:helium_anomaly}, shows the similar behavior of the decay energy of heavy Z=2 and Z=1 isotopes. 
Moreover, one may see that $^{8}$He is more bound than $^{6}$He, which brings the suggestion of small decay energies of the desired $^{7}$H and makes the this task more intriguing.
	%-------------------------------------------------------------------------------
\begin{figure}[t]
	\begin{center}
		\includegraphics[width=0.49\textwidth]{figures/anomalie.png}
	\end{center}
	%
	\caption{Helium-hydrogen anomaly}
	%
	\label{fig:helium_anomaly}
\end{figure}
%-------------------------------------------------------------------------------

Calculations using the seven-body hyperspherical functions formalism \cite{Timofeyuk:2002} evaluated the $^{7}$H g.s.\ energy as $E_{T} \approx 0.84$\,MeV.
But at the same time, based on the assumption, that four valence neutrons of $^{7}$H occupy the same orbitals as in $^{8}$He, the simple estimations of the binding energy of the $^{7}$H ground state were performed in Ref.\ \cite{Korsheninnikov:2003}. 
The obtained value turned out to be $\approx 5.4$\,MeV, which means that this resonance state is expected at about 3\,MeV above the $^3$H+$4n$ decay threshold.
The authors also emphasized that the $^{7}$H ground state should undergo the unique five-body decay into $^3$H+$4n$ with very small width. 
Although, the experimental results of Ref.\ \cite{Korsheninnikov:2003} did not give a chance to identify any suggested states in a structure of $^{7}$H, the evidence of the $^{7}$H ground state resonance near the $^3$H+$4n$ decay threshold was obtained for the first time.
The MM spectrum of $^{7}$H obtained in that work showed a sharp increase starting from the $^3$H+$4n$ threshold.
This observation was important step towards solving the $^{7}$H problem, it did not allow the authors to give quantitative information about the resonance parameters because of low energy resolution (of $\approx 2$\,MeV) and complicated background conditions.

One should also mention the phenomenological estimates in Ref.\ \cite{Golovkov:2004} pointed to $E_{T} \approx 1.3-1.8$\,MeV.
The early pioneering theoretical works suffered from the low computation power and that is why mostly provided low quality estimations by kind of extrapolation methods.
This is clearly seen in works \cite{Aoyama:2004} and \cite{Aoyama:2009} where the same approach provided $E_{T} \approx 7$ and $E_{T} \approx 4$\,MeV respectively.
The authors used so-called antisymmetrized molecular dynamics and gave analysis of the dineutron correlations in heavy helium and hydrogen isotopes.

The modern attempts of search for the superheavy hydrogen isotopes in the reaction of stopped pion absorbtion by light nuclei, are described in works Ref.\ \cite{Gurov:2007,Gurov:2009}.
In contrast to the early works \cite{Seth:1981,Evseev:1981}, the new approach was aimed to observe the short lived neutron rich resonances.
The missing mass method was used for reconstruction of the states of interest from the supposed reaction products.
The authors also conducted multiple correlation measurements of the same reaction mechanism, which allowed to ensure the applicability of this technique for neutron rich light isotopes and to determine the energy resolution of the obtained MM spectra.
One searched $^7$H in the reactions $^{9}$B($\pi^-$,$p$$p$)$^7$H and $^{11}$B($\pi^-$,$p^3$He)$^7$H.
The count rate of the $p$+$^3$He products emitted in the $^{11}$B($\pi^-$,$p^3$He)$^7$H reaction was very low, which allowed only to observe the sharp rising of the MM spectrum near the $^3$H+$4n$ threshold.
Although such results indicated the existence of $^7$H as a resonance near the threshold of the supposed five-body breakup, the authors concluded that the $^{7}$H issue remains open.


\newpage

The $^{7}$H existence was investigated by the authors of Refs.\ \cite{Caamano:2007,Caamano:2008} in the transfer reaction $^{12}$C($^{8}$He,$^{13}$N)$^{7}$H.
Although in this work only seven events could be attributed to the desired reaction channel, a very narrow $^7$H resonance was announced, with $E_T= 0.57^{+0.42}_{-0.21}$\,MeV.
It should be pointed out that no actual reaction channel identification was possible in this experiment.
The interpretation is essentially based on the assumption that only the $^{7}$H g.s.\ is populated in this reaction.
In reality, the population of $^{7}$H$^*$ is also possible in this experiment.
In addition, the reactions $^{12}$C($^{8}$He,$^{14}$N)$^{6}$H and $^{12}$C($^{8}$He,$^{15}$N)$^{5}$H may mock up the detection of $^{7}$H.

The authors of Ref.\ \cite{Fortier:2007} investigated the $^2$H($^8$He,$^3$He)$^7$H reaction.
They concluded that there was some indication of a $^7$H resonance state in the measured MM spectrum at $E_T \approx 2$\,MeV.
It is notable, however, that the experimental acceptance covered only the energies up to 5\,MeV in the $^{7}$H excitation spectrum.
Within this narrow energy window, the $^{7}$H spectrum from the $^2$H($^8$He,$^3$He)$^7$H reaction looks very similar to the spectrum of the  carbon-induced background from the CD$_2$ target, which made the authors cautious about their observations. (Here and in the following, D$_2$ denotes
$^2$H$_2$.)

The next attempt to discover $^{7}$H  using the $^2$H($^8$He,$^3$He)$^7$H reaction was made in Ref.\ \cite{Nikolskii:2010} at RIKEN.
No indication on the resonance peak was revealed in the measured $^7$H MM spectrum.
However, some peculiarity was found in this spectrum at $\approx 2$\,MeV above the $^3$H+$4n$ decay threshold.
The authors reported a value of about 30 $\mu$b/sr in c.m.\ for the cross-section of the reaction populating the low-energy part in the $^7$H spectrum.
In addition, they noted that the $^7$H spectrum demonstrates a peculiarity at about 10.5\,MeV that could be a manifestation of a $^{7}$H continuum excitation.

Для лёгких систем, прямые реакции - самые вероятные , и более того, чтобы достичь сильной ассиметрии массы и заряда, выбивание даже одного нуклона из RIB может быть достаточно получения системы за границей стабильности.

взять начало у Вратислава

Описать 8He, геливую аномалию...