\section{Light neutron-rich exotic nuclear studies}

Search for limits of existence of unbound neutron-rich systems is one of the main trends in modern nuclear physics.
Probably, the most significant recent results in this field are works on 10He [1–4], 13Li [5,6], 16Be [7], 21B [8], and 26O [9–11] and the on going quest for 18Be, 28O, and 33F [12] (and analogous very
exotic species). 
One can find that these experiments require extreme efforts, often leading to poor data quality (statistics, resolution) and, in turn, to numerous unresolved questions and controversies, see, e.g., Refs. [13,14].
The typical feature of the mentioned nuclides is the multineutron (at least two-neutron) emission, and  isotopes  with its four-neutron decay channel represents a very important guideline case for the prospective in this field.
Among all the neutron rich nuclear resonances, 6,7H being the isotopes of the lightest chemical element, draw the attention of physicists because of their biggest neutron-to-proton ratio which can be imagine.

At the present moment, the 7H system is closest to the neutron matter isotope, which makes it the golden fleece for researchers. 
The first theoretical estimations of Baz’ and coworkers [19] predicted that the 7H nucleus could be bound. 
However, the experiments [20,21] searching for 7H formed in the
7Li($\pi-$,$\pi+$) reaction gave negative results. 
Also, the experiment [22] aimed to detect this nucleus among the ternary
fission products of 252 Cf provided no evidence. 
The observation of the ground-state resonance in 5H [15] revived theoretical interest to the possible existence of a low-lying 7H state near the 3H +4n decay threshold. 
Calculations using the seven-body hyperspherical functions formalism [23] evaluated the 7H ground-state energy as E T ~ 3 MeV. 
In Ref. [24] the binding energy of the 7H ground state was estimated to be ~ 5.4 MeV, which means that this resonance state is expected at about 3 MeV above the 3H +4n decay threshold. 
The authors emphasized that the 7H ground state should undergo the unique five-body decay into 3 H +4n with very small width. 
The phenomenological estimates in Ref. [25] pointed to E T ~ 1.3-1.8 MeV. 
The calculations within antisymmetrized molecular dynamics [26,27] provided E T ~ 7 and E T ~ 4 MeV, respectively.

The first experimental evidence of the 7H ground-state resonance was observed in the study of the 1H (8He,2p)7H reaction in Ref. [24]. 
The MM spectrum of 7 H obtained in that work showed a sharp increase starting from the
3H +4n threshold. 
Nevertheless, this interesting observation did not allow the authors to give quantitative information about the resonance parameters because of low energy resolution (of ~2 MeV) and complicated background
conditions.

A sophisticated approach was used in the work [25] carried out by the ACCULINNA fragment-separator group. 
By bombarding a very thick (5.6-cm) liquid deuterium target with a beam of 20.6 A MeV 8He projectiles, the authors searched for the quasistable 7H nuclei produced in the 2H(8He,7H)3He reaction within 0-50 center-of-mass (c.m.) angular range and with a lifetime longer than 1 ns. 
No 7 H events with such lifetime were found. 
This gives a very low limit for the cross section of the 2H(8He,7H)3He reaction, $\sigma$< 3 nb/sr, which is by several orders of the magnitude less than the expected value. 
The lifetime estimates made in Ref. [25] led to the conclusion that the obtained limit of the 7H production cross section implies a lower limit of ET ~ 50–100 keV for its decay energy. 
This indicates that the only realistic approach to the 7 H problem is the search for the shorter-lived resonance states of this nucleus in the five-body 3H +4n continuum.


Для лёгких систем, прямые реакции - самые вероятные , и более того, чтобы достичь сильной ассиметрии массы и заряда, выбивание даже одного нуклона из RIB может быть достаточно получения системы за границей стабильности.

взять начало у Вратислава

Описать 8He, геливую аномалию...