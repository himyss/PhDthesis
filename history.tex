\section{Light neutron-rich exotic nuclear studies}

Search for limits of existence of unbound neutron-rich systems is one of the main trends in modern nuclear physics.
Probably, the most significant recent results in this field are works on 10He [1–4], 13Li [5,6], 16Be [7], 21B [8], and 26O [9–11] and the on going quest for 18Be, 28O, and 33F [12] (and analogous very
exotic species). 
One can find that these experiments require extreme efforts, often leading to poor data quality (statistics, resolution) and, in turn, to numerous unresolved questions and controversies, see, e.g., Refs. [13,14].
The typical feature of the mentioned nuclides is the multineutron (at least two-neutron) emission, and  isotopes  with its four-neutron decay channel represents a very important guideline case for the prospective in this field.
Among all the neutron rich nuclear resonances, 6,7H being the isotopes of the lightest chemical element, draw the attention of physicists because of their biggest neutron-to-proton ratio which can be imagine.

тут про 7Н, потом уже про 6Н....

One may say that the 7H system constitutes the golden fleece for physicists studying light nuclei. 
7H has the maximum neutron excess which can be imagined. 
Thus, the neutron-to-proton ratio amounts to 6/1. The prospects for the search for this nucleus as a narrow low-lying resonance and its importance for the nuclear structure theory were justified by Ya B Zel'dovich [77] at the end of 1950s. 
However, after more than half a century this problem remains unresolved.
Indications that the 7 H ground state resides above the 3H-4n threshold were obtained in the 8He(p,2p) and 11Be($\pi^-$p$^3$He) reactions reported in Refs [186, 187]. 
In neither case did the low experimental resolution (>1 MeV) and bad background conditions allow a quantitative statement about the 7 H properties. 
The discovery of 7 H was declared by Caamano et al. [188] for the 12C(8He,13N) reaction. 
However, a close examination of their detailed paper [189] shows that identification of a final state in these studies is actually absent, and observed events can refer to other hydrogen isotopes with other energies.


Для лёгких систем, прямые реакции - самые вероятные , и более того, чтобы достичь сильной ассиметрии массы и заряда, выбивание даже одного нуклона из RIB может быть достаточно получения системы за границей стабильности.

взять начало у Вратислава

Описать 8He, геливую аномалию...