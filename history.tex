\section{Light neutron-rich exotic nuclear studies}

Search for limits of existence of unbound neutron-rich systems is one of the main trends in modern nuclear physics.
Probably, the most significant recent results in this field are works on $^{10}$He \cite{Sidorchuk:2012,Kohley:2012,Jones:2015,Matta:2015}, $^{13}$Li \cite{Johansson:2010a,Kohley:2013b}, $^{16}$Be \cite{Spyrou:2012}, $^{21}$B \cite{Leblond:2018}, $^{26}$O \cite{Kohley:2013,Caesar:2013,Kondo:2016} and the on-going quest for $^{18}$Be, $^{28}$O, $^{33}$F \cite{Ahn:2019} (and analogous very exotic species). 
One can find that these expements require extreme efforts, often leadinging to poor data quality (statistics, resolution) and, in turn, to numerous unresolved questions and controversies, see, e.g.\ \cite{Grigorenko:2016,Fortune:2018}.
The typical feature of the mentioned nuclides is the multineutron (at least two-neutron) emission and $^{7}$H with its four-neutron decay channel represents a very important guideline case for the prospective studies in this field.
Among all the neutron rich nuclear resonances, 6,7H being the isotopes of the lightest chemical element, draw the attention of physicists because of their biggest neutron-to-proton ratio which can be imagine.

At the present moment, the $^{7}$H system is closest to the neutron matter isotope, which makes it the golden fleece for researchers. 
The first theoretical estimations of Baz' and coworkers \cite{Baz:1972} predicted that the $^{7}$H nucleus could be bound.
However, the experiments \cite{Seth:1981,Evseev:1981} searching for $^{7}$H formed in the $^7$Li($\pi^-,\pi^+$) reaction gave negative results.
Even much later, another sophisticated approach was used in the work \cite{Golovkov:2004}, dedicated to search the long-lived quasistable $^{7}$H and carried out by the ACCULINNA fragment-separator group.
By bombarding a very thick (5.6 cm) liquid deuterium target with a beam of 20.6\,AMeV $^{8}$He  projectiles, the authors searched for the quasistable $^{7}$H nuclei produced in the $^2$H($^8$He,$^7$H)$^3$He reaction within $0^{\circ}-50^{\circ}$ c.m.\ angular range and with such a lifetime longer than 1\,ns.
No $^7$H events with such lifetime was found.
This gives a very low limit for the cross section of the $^2$H($^8$He,$^7$H)$^3$He reaction, $\sigma < 3$ nb/sr, which is by several orders of the magnitude less than the expected value.
The lifetime estimates made in Ref.\ \cite{Golovkov:2004} led to the conclusion that the obtained limit of the $^{7}$H production cross section implies a lower limit of $E_T \gtrsim 50-100$ keV for its decay energy.
This indicates that the only realistic approach to the $^7$H problem is the search for the shorter-lived resonance states of this nucleus in the five-body $^{3}$H+$4n$ continuum.
Also, the experiment \cite{Aleksandrov:1982} aimed to detect this nucleus among the ternary fission products of $^{252}$Cf provided no evidence.
The observation of the ground state resonance in $^{5}$H \cite{Korsheninnikov:2001} revived theoretical interest to the possible existence of a low-lying $^{7}$H  state near the $^{3}$H+$4n$ decay threshold.
Calculations using the seven-body hyperspherical functions formalism \cite{Timofeyuk:2002} evaluated the $^{7}$H g.s.\ energy as $E_{T} \approx 3$\,MeV.
In Ref.\ \cite{Korsheninnikov:2003} the binding energy of the $^{7}$H ground state was estimated to be $\approx 5.4$\,MeV, which means that this resonance state is expected at about 3\,MeV above the $^3$H+$4n$ decay threshold.
The authors emphasized that the $^{7}$H ground state should undergo the unique five-body decay into $^3$H+$4n$ with very small width. 
The phenomenological estimates in Ref.\ \cite{Golovkov:2004} pointed to $E_{T} \approx 1.3-1.8$\,MeV.
The calculations within antisymmetrized molecular dynamics \cite{Aoyama:2004} and \cite{Aoyama:2009} provided $E_{T} \approx 7$ and $E_{T} \approx 4$\,MeV, respectively.

The first experimental evidence of the $^{7}$H g.s.\ resonance was observed in the study of the $^1$H($^8$He,$2p)^7$H reaction in Ref.\  \cite{Korsheninnikov:2003}.
The MM spectrum of $^{7}$H obtained in that work showed a sharp increase starting from the $^3$H+$4n$ threshold.
Nevertheless, this interesting observation did not allow the authors to give quantitative information about the resonance parameters because of low energy resolution (of $\approx 2$\,MeV) and complicated background conditions.

Results obtained in the study of stopped $\pi^-$ absorption by the $^9$Be and $^{11}$B targets were reported in Ref.\ \cite{Gurov:2007}.
The count rate of the $p$+$^3$He products emitted in the $^{11}$B($\pi^-$,$p^3$He)$^7$H reaction was very low.
The authors concluded that the question of the possible existence of the $^{7}$H states, both near the $^3$H+$4n$ threshold and in the region of higher excitation energy remains open \cite{Gurov:2009}.

The $^{7}$H existence was investigated by the authors of Refs.\ \cite{Caamano:2007,Caamano:2008} in the transfer reaction $^{12}$C($^{8}$He,$^{13}$N)$^{7}$H.
Although in this work only seven events could be attributed to the desired reaction channel, a very narrow $^7$H resonance was announced, with $E_T= 0.57^{+0.42}_{-0.21}$\,MeV.
It should be pointed out that no actual reaction channel identification was possible in this experiment.
The interpretation is essentially based on the assumption that only the $^{7}$H g.s.\ is populated in this reaction.
In reality, the population of $^{7}$H$^*$ is also possible in this experiment.
In addition, the reactions $^{12}$C($^{8}$He,$^{14}$N)$^{6}$H and $^{12}$C($^{8}$He,$^{15}$N)$^{5}$H may mock up the detection of $^{7}$H.

The authors of Ref.\ \cite{Fortier:2007} investigated the $^2$H($^8$He,$^3$He)$^7$H reaction.
They concluded that there was some indication of a $^7$H resonance state in the measured MM spectrum at $E_T \approx 2$\,MeV.
It is notable, however, that the experimental acceptance covered only the energies up to 5\,MeV in the $^{7}$H excitation spectrum.
Within this narrow energy window, the $^{7}$H spectrum from the $^2$H($^8$He,$^3$He)$^7$H reaction looks very similar to the spectrum of the  carbon-induced background from the CD$_2$ target, which made the authors cautious about their observations. (Here and in the following, D$_2$ denotes
$^2$H$_2$.)

The next attempt to discover $^{7}$H  using the $^2$H($^8$He,$^3$He)$^7$H reaction was made in Ref.\ \cite{Nikolskii:2010} at RIKEN.
No indication on the resonance peak was revealed in the measured $^7$H MM spectrum.
However, some peculiarity was found in this spectrum at $\approx 2$\,MeV above the $^3$H+$4n$ decay threshold.
The authors reported a value of about 30 $\mu$b/sr in c.m.\ for the cross-section of the reaction populating the low-energy part in the $^7$H spectrum.
In addition, they noted that the $^7$H spectrum demonstrates a peculiarity at about 10.5\,MeV that could be a manifestation of a $^{7}$H continuum excitation.

Для лёгких систем, прямые реакции - самые вероятные , и более того, чтобы достичь сильной ассиметрии массы и заряда, выбивание даже одного нуклона из RIB может быть достаточно получения системы за границей стабильности.

взять начало у Вратислава

Описать 8He, геливую аномалию...