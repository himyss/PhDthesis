\section{Introduction}

The subject of nuclear physics investigations is correlated many-body system of fermions. 
Although most of all one was focused on study two-component systems, consisted of neutrons and protons (nucleons), the limits of such investigations are very far by the present moment.
However, more than three thousand proton-neutron configurations, isotopes, have been already	 discovered.
And one cannot speak about the exact number of them, because it is continuously growing. 
The theoretical estimation predict that from 2000 to 3000 other nuclear systems can be synthesized.  
All the nuclei can be classified by the number of nucleons (A). 
In light (A<12) stable nuclei, the number of neutrons (N) is approximately equal to the number of protons (Z). 
For the stable heavy systems, the ratio of mass over charge is increasing and reaches the value of 1.6 in a region of A>250. 
This neutron-proton asymmetry for is explained by the very short range of nuclear forces and Coulomb repulsion.

It is known that there are 243 stable nuclei, but in nature one can find about 339 \textcolor{red}{[cite UFN OLD]}. 
That means, that some of them are unstable with respect to $\alpha-$, $\beta-$ or $\gamma-$decay and fission (for heavy elements), but they are still bound and live long enough to come down to us from the early universe, or are decay products of heavier nuclear systems. 
Bound nuclear systems are defined by mean of single or multiple nucleon separation energy.
As further the isotope is located on from the line of stability \textcolor{red}{REF CHART}, the less is the separation energy.
At some point for such system it becomes energetically favorable to emit one or several nucleons.
At this point the border between bound and unbound nuclear systems, called dripline.
One can also notice that there are many driplines, related to different decay processes.
Beyond the driplines nuclear systems can exist as so-called resonances. 
All resonances, together with exotic nuclei, located in the vicinity of the driplines, are called exotic nuclear systems.
Usually, as more exotic the system is, the less its lifetime, which makes studies of exotic states challenging. 
That is why the location of the driplines was experimentally obtained only for the light nuclei (for systems with Z<32 or N<20) \textcolor{red}{[cite UFN DERICA 2019]}.
And moreover, so far there is no answer to a fundamental question: how far beyond the dripline can the resonances be located? 
And moreover, it remains an open important fundamental question as to whether the border between unbound nuclear systems and continuous spectra is existed, and where is it located.

Aside from the experimental issues of studies of short-lived exotic nuclear systems, there are no universal theoretical models, describing the nuclear matter under such extreme conditions.
The reason for this is that exotic systems differ from the nuclei located in a line of stability not only in Z/A ration but also in unique phenomena existence.
Halo and skin effects, different magic shell numbers, new types of excitation and radioactivity make most of the nuclear structure models inapplicable for description of such unstable systems.

\begin{itemize}
	\item 
	One can suggest that in exotic nuclear systems, those valence nucleons, occupied the highest shells, can tunell to the classically forbidden regions. 
	The manifestations of such phenomenom are the different spatial distribution of the valence nucleons, compared to others. 
	That is why, such transformed systems have larger corresponded radii than their stable isotopes.
	One can imagine such system as cluster-core, surrounded by the valence nucleons with smaller separation energy.
	For the fist time, the term "halo" was used in the work \textcolor{red}{[B. Jonson P.G. Hansen. The Neutron Halo of Extremely Neutron-Rich Nuclei. Europhys. Lett., 4(4):409–414, 1987.]}, but by the present day, it is well known, that such exotic feature was observed in many neutron-rich (6He, 11Li,17B, 19B, 22C) and neutron-deficient systems (8B, 13N, 17Ne, 26P, 27S) \textcolor{red}{[some cite to the HALO summary report]}.
	In the systems with neutron excess, so-called "neutron skin" phenomenon can be also observed.
	Such systems are characterized by the extension of size, but not the form. 8He and 14Be are well known examples of the isotopes with neutron skin.
	
	\item 
	The spatial transformations of nucleons lead to the new freedom degree, so-called soft excitation mode, which can be described as oscillations of the valence nucleons with respect to the core cluster. 
	
	\item 
	Many body decays, borromian nuclei. 

\end{itemize}



Studies of the light exotic nuclear systems take a particular significant role in modern physics.
The limited number of nucleons of light isotopes (A<20) allows to highlight the contribution of the studied phenomena effects among other processes, which is extremely convenient for reactions of exotic system production, characterized by low cross sections.
On top of that, in the light isotopes the extrema of ratio of mass-over-charge can be reached.
These short-lived systems are usually characterized by many-body decay channels.

Investigation of such simultaneous many-body decays is the only way to get experimental information about the many-body capture processes, which can be considered as the time reversal reactions to the observed decays.
These processes may occur only in under extreme conditions of very high temperature and density, that is why investigations of exotic nuclear systems may have a significant contribution to astrophysics.
%Both fast mechanisms of nucleosynthesis (r- and rp-processes), occurred in compact objects with isotopes, located nearby both proton and neutron driplines. 
For example, investigations of the neutron-deficient systems in the vicinity of the proton dripline can be an instrument to study the nucleosynthesis of fast proton-capture, so-called rp-process, occurred on the surface of a neutron stars in a reactions of hydrogen burning \textcolor{red}{[UFN 2019, [Grigorenko L V Phys. Part. Nucl. 40 674 (2009); Fiz. Elem. Chast. At. Yad. 40 1271 (2009)],  Grigorenko L V et al. Phys. Lett. B 641 254 (2006),  Grigorenko L V, Zhukov M V Phys. Rev. C 72 015803 (2005)]}. 
The features of this rp-process at waiting points, at which single proton capture is forbidden, can be explored using the reactions of multi-proton decays \textcolor{red}{[ Gorres J, Wiescher M, Thielemann F-K Phys. Rev. C 51 392 (1995)]}.
On the other hand, the rapid neutron-capture process, also known as the r-process, proceeds along the neutron dripline in supernova core-collapse processes, is similarly studied in reactions with neutron-rich isotopes.
Therefore, investigation of multi-nucleon emitters is a good test for developing theoretical models, which are supposed to provide the equation of state of the nuclear matter at extreme conditions. 

This thesis is devoted to studies of light neutron-rich exotic nuclear systems.
The heavy neutron excess isotopes of the first four chemical elements allow to synthesize the system with the biggest ratio of mass over charge, approaching the neutron matter study.
This huge unique asymmetry allows different exotic features described above to be observed.










