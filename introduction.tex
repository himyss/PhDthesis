\section{Introduction}

The subject of nuclear physics investigations is correlated many-body system of fermions. 
Although most of all one was focused on study two-component systems, consisted of neutrons and protons (nucleons), the limits of such investigations are very far by the present moment.
However, more than three thousand proton-neutron configurations, isotopes, have been already discovered.
And one cannot speak about the exact number of them, because it is continuously growing. 
The theoretical estimation predict that from 2000 to 3000 other nuclear systems can be synthesized.  
All the nuclei can be classified by the number of nucleons (A). 
In light (A<12) stable nuclei, the number of neutrons (N) is approximately equal to the number of protons (Z). 
For the stable heavy systems, the ratio of mass over charge is increasing and reaches the value of 1.6 in a region of A>250. 
This neutron-proton asymmetry for is explained by the very short range of nuclear forces and Coulomb repulsion.

It is known that there are 243 stable nuclei, but in nature one can find about 339 \textcolor{red}{[cite UFN OLD]}. 
That means, that some of them are unstable with respect to $\alpha-$, $\beta-$ or $\gamma-$decay and fission (for heavy elements), but they are still bound and live long enough to come down to us from the early universe, or are decay products of heavier nuclear systems. 
Bound nuclear systems are defined by mean of single or multiple nucleon separation energy.
As further the isotope is located on from the line of stability \textcolor{red}{REF CHART}, the less is the separation energy.
At some point for such system it becomes energetically favorable to emit one or several nucleons.
At this point the border between bound and unbound nuclear systems, called dripline.
One can also notice that there are many driplines, related to different decay processes.
Beyond the driplines nuclear systems can exist as so-called resonances. 
All resonances, together with exotic nuclei, located in the vicinity of the driplines, are called exotic nuclear systems.
Usually, as more exotic the system is, the less its lifetime, which makes studies of exotic states challenging. 
That is why the location of the driplines was experimentally obtained only for the light nuclei (for systems with Z<32 or N<20) \textcolor{red}{[cite UFN DERICA 2019]}.
And moreover, so far there is no answer to a fundamental question: how far beyond the dripline can the resonances be located? 
And moreover, it remains an open important fundamental question as to whether the border between unbound nuclear systems and continuous spectra is existed, and where is it located.

Aside from the experimental issues of studies of short-lived exotic nuclear systems, there are no universal theoretical models, describing the nuclear matter under such extreme conditions.
The reason for this is that exotic systems differ from the nuclei located in a line of stability not only in Z/A ration but also in unique phenomena existence.
Halo and skin effects, different magic shell numbers, new types of excitation and radioactivity make most of the nuclear structure models inapplicable for description of such unstable systems.

\begin{itemize}
	\item 
	One can suggest that in exotic nuclear systems, those valence nucleons, occupied the highest shells, can tunell to the classically forbidden regions. 
	The manifestations of such phenomenom are the different spatial distribution of the valence nucleons, compared to others. 
	That is why, such transformed systems have larger corresponded radii than their stable isotopes.
	One can imagine such system as cluster-core, surrounded by the valence nucleons with smaller separation energy.
	It leads not only to different proton and neutron radii, but other anomalies, such as reaction cross section increasing, very narrow momentum distribution of the valence nucleons, etc.
	For the fist time, the term "halo" was used in the work \textcolor{red}{[B. Jonson P.G. Hansen. The Neutron Halo of Extremely Neutron-Rich Nuclei. Europhys. Lett., 4(4):409–414, 1987.]}, but by the present day, it is well known, that such exotic feature was observed in many neutron-rich (6He, 11Li,17B, 19B, 22C) and neutron-deficient systems (8B, 13N, 17Ne, 26P, 27S) \textcolor{red}{[some cite to the HALO summary report]}.
	In the systems with neutron excess, so-called "neutron skin" phenomenon can be also observed.
	Such systems are characterized by the extension of size, but not the form. 8He and 14Be are well known examples of the isotopes with neutron skin.
	
	\item 
	The spatial transformations of nucleons lead to the new freedom degree, so-called soft excitation mode, which can be described as oscillations of the valence nucleons with respect to the core cluster. 
	
	\item 
	Some core-halo nuclear systems may represent so-called "Borromean nuclei", which were named after the Borromean rings - the mathematical object, consisted of three topologically licked curves, which cannot be separated, but not without any binary links between each other.
	Removing any part of it, one disintegrate the whole system.
	In other words, Borromean nuclei are bound only as many-body system, while all its subsystems are unbound.
	This has led to simultaneous many-body decays of some exotic nuclear systems.
	

\end{itemize}



Studies of the light exotic nuclear systems take a particular significant role in modern physics.
The limited number of nucleons of light isotopes (A<20) allows to highlight the contribution of the studied phenomena effects among other processes, which is extremely convenient for reactions of exotic system production, characterized by low cross sections.
On top of that, in the light isotopes the extrema of ratio of mass-over-charge can be reached.
These short-lived systems are usually characterized by many-body decay channels.

Investigation of such simultaneous many-body decays is the only way to get experimental information about the many-body capture processes, which can be considered as the time reversal reactions to the observed decays.
These processes may occur only in under extreme conditions of very high temperature and density, that is why investigations of exotic nuclear systems may have a significant contribution to astrophysics.
%Both fast mechanisms of nucleosynthesis (r- and rp-processes), occurred in compact objects with isotopes, located nearby both proton and neutron driplines. 
For example, investigations of the neutron-deficient systems in the vicinity of the proton dripline can be an instrument to study the nucleosynthesis of fast proton-capture, so-called rp-process, occurred on the surface of a neutron stars in a reactions of hydrogen burning \textcolor{red}{[UFN 2019, [Grigorenko L V Phys. Part. Nucl. 40 674 (2009); Fiz. Elem. Chast. At. Yad. 40 1271 (2009)],  Grigorenko L V et al. Phys. Lett. B 641 254 (2006),  Grigorenko L V, Zhukov M V Phys. Rev. C 72 015803 (2005)]}. 
The features of this rp-process at waiting points, at which single proton capture is forbidden, can be explored using the reactions of multi-proton decays \textcolor{red}{[ Gorres J, Wiescher M, Thielemann F-K Phys. Rev. C 51 392 (1995)]}.
On the other hand, the rapid neutron-capture process, also known as the r-process, proceeds along the neutron dripline in supernova core-collapse processes, is similarly studied in reactions with neutron-rich isotopes.
Therefore, investigation of multi-nucleon emitters is a good test for developing theoretical models, which are supposed to provide the equation of state of the nuclear matter at extreme conditions. 

\subsection{Experimental techniques for exotic nuclear research}

Due to the short lifetime, exotic nuclei can not be found in nature.
And even with modern technologies, studies of such abnormal systems is a great technical challenge due to low cross sections of the reactions of interest
%Different reaction mechanisms were used for the new isotopes production: inelastic scattering of stable light and heavy beams at various energies, 
New isotopes can be products of different types of reactions.
\begin{itemize}
	\item 
	\textbf{Pickup or stripping transfer reactions} are widely used mechanisms, especially for light system research, in which the beam exchanges the nucleons with a beam. 
	Stripping is transfer from the projectile to the target, and pickup from the target to the projectile.
	
	\item 
	In the \textbf{fusion reactions} the system with N-Z combination, approximately equal to the sum of the number of protons and neutrons in the initial beam and target nuclei.
	Due to the increasing asymmetry of mass over charge for heavy nuclei, such reaction mechanism is commonly used in studies of neutron-deficient isotopes.	
	In the subsequent decays or fission processes of the compound nucleus, one also can produce the systems of interest. 
	
	\item
	In the knock-out reactions, the isotopes can be produced in a single nucleon or a light cluster removal process from the projectile by a collision with the target.
	
	\item
	In order to obtain exotic nuclei, the so-called recharging reactions can be applied.
	In these reactions predominantly  the ground state of exotic product is populated by replacing of one or few protons (neutrons) with neutrons (protons).
	
	
\end{itemize}	

\textcolor{red}{тут написать что всё это уже исчерпало себя если использовать стабильные пучки.}

%Eventually, over time, most of the used methods of production nuclear states close and beyond the dripline   


The most widely applied 
At the present moment, most of the exotic isotopes, were discovered in the experiments with radioactive ion beams (RIB).


В лабораторных условиях получать ядра вблизи границы стабильности сложно из-за малых сечений образования этих ядер и коротких периодов полураспада. В настоящее время методы сепарации и детектирования образующихся в результате ядерных реакций экзотических ядер достигли такого уровня, что основные характеристики атомных ядер: масса, период полураспада, основные моды распада - могут быть получены на основе анализа небольшого числа ядер(например \cite{flnr}).

Одним из способов изучения экзотических ядер являются эксперименты с использованием радиоактивных пучков. Проблема получения таких пучков заключается в том, что эти ядра, как правило,  имеют короткие (менее 1 секунды) времена жизни, что делает невозможным использование традиционных методов, применяемых для получения пучков стабильных ядер. В связи с этим возникла необходимость разработки специальных методов выделения требуемого изотопа из спектра вторичных частиц, который образуется в результате взаимодействия первичного пучка с мишенью.

Существует два основных метода получения пучков радиоактивных ядер.
Исторически первым методом работы с пучками радиоактивных изотопов стал метод ISOL (Isotope Separation On-Line)\cite{ufn}. Ядро-мишень разрушается в реакции расщепления лёгкими налетающими частицами. Мишень сделана таким образом, что продукты реакции застревают в её объеме. Далее, нужные изотопы определенным способом извлекаются из мишени, проходят сепарацию и ионизацию. После сепарации пучок радиоактивных частиц имеет кинетическую энергию в диапазоне от 10 до 500 КэВ и может быть использован в экспериментах с низкими энергиями, или может быть направлен в ускоритель вторичных частиц, если время жизни ядер это позволяет. В последнем случае качество вторичного пучка определяется параметрами соответствующего ускорителя и не отличается от качества стандартных пучков стабильных ядер. Времена извлечения радиоактивных ядер из мишени и их транспортировка ко второму ускорителю определяют диапазон времен жизни экзотических ядер, для которых может быть использован этот метод. Современные ISOL-методы уверенно обеспечивают время выделения порядка 100мс\cite{ufn}. 
%Преимущество ISOL состоит в том, что этот метод не требует дополнительной диагностики свойств пучка.

Значительными преимуществами по времени выделения и интенсивности пучков радиоактивных изотопов обладает метод разделения на лету (In-Flight Separation). В этом методе пучок частиц с энергией от 30 МэВ/нуклон до 1 ГэВ/нуклон налетает на производящую мишень (в основном бериллиевую). В результате столкновения получается большое количество осколков-фрагментов, которые летят преимущественно по направлению первичного пучка и попадают во фрагмент-сепаратор, выделяющий частицы с определенным значением магнитной жесткости($\xi$), которая определяется отношением :

\begin{equation}
\label{Mag}
\xi = B \rho =  \frac{pc}{Ze},
\end{equation}

где $p$ - импульс частицы, $Z$ — атомный номер, $c$ – скорость света, $e$ – элементарный электрический заряд, $\rho$ - радиус кривизны траектории движения частицы, $B$ - вектор магнитной индукции.  На выходе фрагмент-сепаратора, в зависимости от степени очистки, помимо интересующего нас изотопа, мы имеем коктейль из примесных изотопов, имеющих близкие значения магнитной жесткости. Энергетическая дисперсия пучка определяется аксептансом фрагмент-сепаратора и составляет, как правило, величину порядка нескольких процентов от полной энергии пучка. 


%Существует два основных метода получения пучков радиоактивных ядер: 
%
%1. ISOL (Isotope Separation On Line). В этом методе, в отличие от In-Flight, процессы производства и ускорения экзотических ядер отделены друг от друга. Мишень сделана таким образом, что продукты реакции застревают в её объеме. Далее, нужные изотопы определенным способом извлекаются из мишени, проходят сепарацию и ионизацию. На выходе из сепаратора пучок радиоактивных частиц имеет энергию в диапазоне от 10 до 500 кэВ и может быть использован в экспериментах с низкими энергиями, или может быть направлен в ускоритель вторичных частиц. В последнем случае качество вторичного пучка определяется параметрами соответствующего ускорителя и не отличается от качества стандартных пучков стабильных ядер. Времена извлечения радиоактивных ядер из мишени и их транспортировка ко второму ускорителю определяют диапазон времен жизни экзотических ядер, для которых может быть использован этот метод. Преимущество ISOL состоит в том, что этот метод не требует дополнительной диагностики свойств пучка.
%
%2. In-Flight. Пучок частиц с энергией от 30 МэВ/нуклон до 1 ГэВ/нуклон налетает на производящую мишень (в основном бериллиевую). В результате столкновения получается большое количество осколков-фрагментов, которые летят преимущественно вперед и попадают во фрагмент-сепаратор, выделяющий частицы с определенным значением магнитной жесткости. На выходе фрагмент-сепаратора, в зависимости от степени очистки, помимо интересующего нас изотопа, мы имеем коктейль из примесных изотопов, имеющих близкие значения магнитной жесткости. Энергетическая дисперсия пучка определяется аксептансом фрагмент-сепаратора и составляет, как правило, величину порядка нескольких процентов от полной энергии пучка. 
%
Для пучков вторичных частиц, полученных более быстрым методом In-Flight, требуется диагностика свойств пучка, таких как энергия, состав и др.



This thesis is devoted to studies of light neutron-rich exotic nuclear systems.
The heavy neutron excess isotopes of the first four chemical elements allow to synthesize the system with the biggest ratio of mass over charge, approaching the neutron matter study.
This huge unique asymmetry allows different exotic features described above to be observed.

Outline of the Thesis
\textcolor{red}{состав, списать у chudoba}

Упомянуть, что в этой работе h==c==1, unit system











