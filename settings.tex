\usepackage[left=2.5cm,right=1.5cm,
top=2cm,bottom=2cm,bindingoffset=0cm]{geometry} % вроде бы поля

\usepackage[utf8]{inputenc} % размер

\usepackage[T2A]{fontenc} % указывает внутреннюю кодировку TeX

\usepackage[ngerman,english,russian]{babel}

\usepackage{amsmath}
\usepackage{mathtools}
\usepackage{mathptmx}
\usepackage{amsfonts}
\usepackage{amssymb}					% поиск в PDF
\usepackage{mathtext} 				% русские буквы в фомулах

\usepackage{epsfig}
\usepackage{xcolor}

\usepackage[outdir=./figures]{epstopdf}

%%% Работа с картинками
\usepackage[labelformat=simple]{subcaption}
\usepackage{array,graphicx,caption}
\usepackage{xcolor,color}		%для наверно цветов

\usepackage{setspace}  % межстрочный интервал
\onehalfspacing

